    \documentclass{book}
    \usepackage[utf8x]{inputenc}
    \usepackage[T2A]{fontenc}
    \usepackage[english,russian]{babel}
    \usepackage[fleqn]{amsmath}

    \usepackage[margin=0.7in,paperwidth=6.1055555in,paperheight=9in]{geometry}


    \usepackage{amssymb,amsmath,amsfonts,latexsym,mathtext}

    \usepackage[usenames,dvipsnames,svgnames,table]{xcolor}
    \setcounter{page}{116}

    \setlength{\parskip}{5pt}



    \usepackage{fancyhdr}
    \pagestyle{fancy}

    \fancyhead[LE]{\thepage}
    \fancyhead[RE]{\small 6 Analogy}

    \fancyhead[RO]{\thepage}
    \fancyhead[LO]{\small 6.4 Tangent roots: A daunting transcendental sum}



    \fancyfoot[L,R,C]{}
    \renewcommand{\headrulewidth}{0pt}
    \usepackage[colorlinks=true, allcolors=blue]{hyperref}

    \usepackage{color}
    \definecolor{light-gray}{rgb}{0.8,0.8,0.8}





    \begin{document}



    \noindent The final, low-entropy sum is the famous Basel sum (high-entropy results
    are not often famous). Its value is B = $\pi^2$/6 (Problem 6.22).\\

    \noindent $\vartriangleright$ \textit{How does knowing} B = $\pi^2$/6 \textit{help evaluate the original sum} $\sum_{1}^\infty (2n+1)^{-2}$\textit{?}\\


    \noindent The major modification from the original sum was to include the even
    squared reciprocals. Their sum is B/4.

    $$
    \sum_{1}^\infty \frac{1}{(2n)^2}=\frac{1}{4}\sum_{1}^\infty \frac{1}{n^2}\eqno(6.26)
    $$
    \noindent The second modification was to include the n = 0 term. Thus, to obtain $\sum_{1}^\infty (2n+1)^{-2}$ , adjust the Basel value B by subtracting B/4 and then the n = 0 term. The result, after substituting B = $\pi^2$/6 , is
    $$
    \sum_{1}^\infty \frac{1}{(2n+1)^2}=B-\frac{1}{4}B-1=\frac{\pi^2}{8}-1\eqno (6.27)
    $$

    \noindent This exact sum, based on the asymptote approximation for $x_n$, produces the following estimate of S.\\


    $$
    S \approx \frac{4}{\pi^2}\sum_{1}^\infty \frac{1}{(2n+1)^2}=\frac{4}{\pi^2} \left( \frac{\pi^2}{8}-1 \right) \eqno (6.28)
    $$
    Simplifying by expanding the product gives
    $$
    S \approx \frac{1}{2}-\frac{4}{\pi^2}=0.094715 \ldots \eqno (6.29)
    $$

    \noindent \colorbox{light-gray}{
    \begin{minipage}{\textwidth}
    \textbf{Problem 6.25} \qquad \textbf{Check the earlier reasoning}\\

    Check the earlier pictorial reasoning (Problem 6.24) that 1/6 + 1/18 = 2/9
    underestimates $\sum_{1}^\infty(2n+1)^{-2}$. How accurate was that estimate?

    \end{minipage}
    }

    \noindent This estimate of S is the third that uses the asymptote approximation\\ $x_n \approx(n+0.5)\pi.$ Assembled together, the estimates are




    $$
    S\approx \begin{cases}
    0.067547 \qquad \text{(integral approximation to $\sum_{1}^\infty (2n+1)^{-2}),$}\\
    0.090063 \qquad \text{integral approximation and triangular overshoots),}\\
    0.094715 \qquad \text{exact sum of $\sum_{1}^\infty (2n+1)^{-2}),$}
    \end{cases}
    $$


    \noindent Because the third estimate incorporated the exact value of $\sum_{1}^\infty (2n+1)^{-2}$,
    any remaining error in the estimate of S must belong to the asymptote
    approximation itself.

    \clearpage
    \newpage


    \noindent $\vartriangleright$ \textit {For which term of $\sum_{} x_n^{-2} $ is the asymptote approximation most inaccurate?}\\

    \noindent As x grows, the graphs of x and tan x intersect ever closer to the vertical
    asymptote. Thus, the asymptote approximation makes its largest absolute
    error when n = 1. Because $x_1$ is the smallest root, the fractional error
    in $x_n$ is, relative to the absolute error in $x_n$, even more concentrated at
    n = 1. The fractional error in $x_n^{-2}$ being −2 times the fractional error
    in $x_n$ (Section 5.3), is equally concentrated at n = 1. Because $x_n^{-2}$ is the
    largest at n = 1, the absolute error in $x_n^{-2}$ (the fractional error times $x_n^{-2}$ itself) is, by far, the largest at n = 1.\\

    \noindent \colorbox{light-gray}{
    \begin{minipage}{\textwidth}
    \textbf{Problem 6.26} \qquad \textbf{Absolute error in the early terms}\\

    Estimate, as a function of n, the absolute error in $x_n^{-2}$ that is produced by the
    asymptote approximation.
    \end{minipage}
    }\\

    \noindent With the error so concentrated at n = 1, the greatest improvement in the
    estimate of S comes from replacing the approximation $x_1 =(n+0.5)\pi$ with a more accurate value. A simple numerical approach is successive
    approximation using the Newton–Raphson method (Problem 4.38). To
    find a root with this method, make a starting guess x and repeatedly
    improve it using the replacement
    $$
    x \longrightarrow x - \frac{\tan{x}-x}{\sec^2{x}-1}\eqno(6.30)
    $$
    \noindent When the starting guess for x is slightly below the first asymptote at 1.5$\pi$, the procedure rapidly converges to $x_1=4.4934 \ldots$\\

    \noindent Therefore, to improve the estimate $S\approx 0.094715$, which was based on the asymptote approximation, subtract its approximate first term (its big part)
    and add the corrected first term.
    $$
    S \approx S_{old} - \frac{1}{(1.5\pi)^2}+\frac{1}{4.4934^2} \approx 0.09921. \eqno (6.31)
    $$
    Using the Newton–Raphson method to refine, in addition, the ${1}/{x_2^2}$ term
    gives $ S \approx 0.09978$ (Problem 6.27). Therefore, a highly educated guess is
    $$
    S=\frac{1}{10}\eqno(6.32)
    $$

    \noindent The infinite sum of unknown transcendental numbers seems to be neither
    transcendental nor irrational! This simple and surprising rational number
    deserves a simple explanation.
    \clearpage
    \newpage

    \noindent \colorbox{light-gray}{
    \begin{minipage}{\textwidth}
    \textbf{Problem 6.27} \qquad \textbf{ Continuing the corrections}\\
    Choose a small N, say 4. Then use the Newton–Raphson method to compute
    accurate values of $x_n$ for n=1 \ldots N; and use those values to refine the estimate
    of S. As you extend the computation to larger values of N, do the refined
    estimates of S approach our educated guess of 1/10?
    \end{minipage}
    }\\
    

    \noindent \textbf{{\large 6.4.3 Analogy with polynomials}}\\

    \noindent If only the equation $\tan{x}-x=0$ had just a few closed-form solutions!
    Then the sum S would be easy to compute. That wish is fulfilled by
    replacing $\tan{x}-x$ with a polynomial equation with simple roots. The
    simplest interesting polynomial is the quadratic, so experiment with a
    simple quadratic — for example, $x^2-3x+2.$\\

    \noindent This polynomial has two roots, $x_1=1$ and $x_2=2;$ therefore $\sum_{}x_n^{-2}$, the
    polynomial-root sum analog of the tangent-root sum, has two terms.
    $$
    \sum_{}x_n^{-2}=\frac{1}{1^2}+\frac{1}{2^2}=\frac{5}{4}.\eqno(6.33)
    $$
    \noindent This brute-force method for computing the root sum requires a solution
    to the quadratic equation. However, a method that can transfer to the
    equation $\tan{x}-x=0$, which has no closed-form solution, cannot use
    the roots themselves. It must use only surface features of the quadratic---namely, its two coefficients 2 and −3. Unfortunately, no plausible method of combining 2 and −3 predicts that $\sum_{}x_n^{-2}=5/4.$\\

    \noindent $\vartriangleright$ \textit {Where did the polynomial analogy go wrong?}\\

    \noindent The problem is that the quadratic $x^2-3x+2$ is not sufficiently similar to $\tan{x}-x$. The quadratic has only positive roots; however, $\tan{x}-x$, an
    odd function, has symmetric positive and negative roots and has a root
    at x = 0. Indeed, the Taylor series for $\tan{x}$ is $x+x^3/3+2x^5/15+\ldots$ (Problem 6.28); therefore,
    $$
    \tan{x}-x=\frac{x^3}{3}+\frac{2x^5}{15}+\ldots \eqno(6.34)$$

    \noindent The common factor of $ x^3$ means that $\tan{x}-x$ has a triple root at x = 0.An analogous polynomial — here, one with a triple root at x = 0, a positive
    root, and a symmetric negative root---is $(x+2)x^3(x-2)$ or, after expansion, $x^5-4x^3$. The sum $\sum_{}x_n^{-2}$(using the positive root) contains only one term
    \clearpage
    \newpage

    \noindent and is simply 1/4. This value could plausibly arise as the (negative) ratio
    of the last two coefficients of the polynomial.\\

    \noindent To decide whether that pattern is a coincidence, try a richer polynomial:
    one with roots at −2, −1, 0 (threefold), 1, and 2. One such polynomial is\\
    $$
    (x+2)(x+1)x^3(x-1)(x-2)=x^7-5x^5+4x^3.\eqno(6.35)
    $$

    \noindent The polynomial-root sum uses only the two positive roots 1 and 2 and is $1/1^2+1/2^2$, which is 5/4 — the the (negative) ratio of the last two coefficients.
    As a final test of this pattern, include −3 and 3 among the roots. The
    resulting polynomial is
    $$
    (x^7-5x^5+4x^3)(x+3)(x-3)=x^9-14x^7+49x^5-36x^3.\eqno(6.36)
    $$

    \noindent The polynomial-root sum uses the three positive roots 1, 2, and 3 and is $1/1^2+1/2^2+1/3^2$, which is 49/36 — again the (negative) ratio of the last
    two coefficients in the expanded polynomial.\\

    \noindent $\vartriangleright$ \noindent \textit{What is the origin of the pattern, and how can it be extended to} $\tan{x}-x$\textit{?}\\

    \noindent To explain the pattern, tidy the polynomial as follows:
    $$
    x^9-14x^7+49x^5-36x^3=-36x^3 \left(1-\frac{49}{36}x^2+\frac{14}{36}x^4-\frac{1}{36}x^6 \right).\eqno(6.37)
    $$

    \noindent In this arrangement, the sum 49/36 appears as the negative of the first
    interesting coefficient. Let’s generalize. Placing k roots at x = 0 and single
    roots at $\pm x_1, \pm x_2, \ldots, \pm x_n$ gives the polynomial
    $$
    Ax^k \left(1-\frac{x^2}{x_1^2}\right)\left(1-\frac{x^2}{x_2^2}\right)\left(1-\frac{x^2}{x_3^2}\right)\ldots\left(1-\frac{x^2}{x_n^2}\right),\eqno(6.38)
    $$
    \noindent where A is a constant. When expanding the product of the factors in
    parentheses, the coefficient of the $x^2$ term in the expansion receives one
    contribution from each $x^2/x_k^2$ term in a factor. Thus, the expansion begins
    $$
    Ax^k \left[1-\left(\frac{1}{x_1^2}+\frac{1}{x_2^2}+\frac{1}{x_3^2}+\ldots+\frac{1}{x_n^2}\right)x^2+\ldots \right].\eqno(6.39)
    $$

    \noindent The coefficient of $x^2$ in parentheses is $\sum_{}x_n^{-2}$, which is the polynomial
    analog of the tangent-root sum.\\

    \noindent Let’s apply this method to $\tan{x}-x$. Although it is not a polynomial, its
    Taylor series is like an infinite-degree polynomial. The Taylor series is

    \clearpage
    \newpage

    $$
    \frac{x^3}{3}+\frac{2x^5}{15}+\frac{17x^7}{315}+\ldots =\frac{x^3}{3} \left(1+\frac{2}{5}x^2+\frac{17}{105}x^4+\ldots \right). \eqno(6.40)
    $$


    \end{document}


